\documentclass{article}
\usepackage[english]{babel}
\usepackage[letterpaper,top=2cm,bottom=2cm,left=3cm,right=3cm,marginparwidth=1.75cm]{geometry}

\usepackage{amsmath}
\usepackage{graphicx}
% add graphics path
\graphicspath{{./graph/}}
\usepackage{indentfirst}
\usepackage{amsfonts}
\usepackage{caption}
\usepackage[colorlinks=true, allcolors=blue]{hyperref}

\usepackage{upgreek}

\title{Assignment 2}
\author{Shuhao Bian}

\begin{document}
\maketitle

\section{Problem 1}

% insert a pdf here
\begin{figure}[h!]
    \centering
    \includegraphics[width=0.7\textwidth]{graph 1.png}
    \caption{Graph 1}\label{fig:pb1graph1}
\end{figure}

According to the Figure \ref{fig:pb1graph1}, the best input is:
%itemize
\begin{itemize}
    \item $ x^*(0)=1\rightarrow x^*(1)=1\rightarrow x^*(2)=0.5, (u^*(1)=0, u^*(2)=-0.5)$
    \item $ x^*(0)=1\rightarrow x^*(1)=0.5\rightarrow x^*(2)=0.5, (u^*(1)=-0.5, u^*(2)=0)$
\end{itemize}
\newpage
\section{Problem 2}
\subsection{No constrants input, $T_s=0.1s$}

\begin{figure}[h!]
    \centering
    \includegraphics[width=0.7\textwidth]{pb1 0.png}
    \caption{Graph q1}\label{fig:pb2graph1}
\end{figure}
In the figure \ref{fig:pb2graph1}, the output of the system followed the
refernce well with no overshoot and no undershoot. The maximum input is
bigger than 10, $\Delta u$ is unlimited.

\subsection{$\left| u(t)\right| \leq 10, T_s=0.1s$}

\begin{figure}[h!]
    \centering
    \includegraphics[width=0.7\textwidth]{pb1 1.png}
    \caption{Graph q2}\label{fig:pb2graph2}
\end{figure}
In the figure \ref{fig:pb2graph2}, the input signal is limited to 10,
the output of the system cannot follow the reference well.

\subsection{$\left| u(t)\right| \leq 10, \left| \Delta u(t) \right| \leq 1$, $T_s=0.1s$}
\begin{figure}[h!]
    \centering
    \includegraphics[width=0.7\textwidth]{pb1 2.png}
    \caption{Graph q3}\label{fig:pb2graph3}
\end{figure}
In the figure \ref{fig:pb2graph3}, the input signal is limited to 10,
and $\Delta u$ is limited to 1, the output of the system cannot follow
the reference well.

\subsection{$\left| u(t)\right| \leq 10, \left| \Delta u(t) \right| \leq 3$, $T_s=0.1s$}

\begin{figure}[h!]
    \centering
    \includegraphics[width=0.7\textwidth]{pb1 3.png}
    \caption{Graph q4}\label{fig:pb2graph4}
\end{figure}
In the figure \ref{fig:pb2graph4}, the input signal is limited to 10,
and $\Delta u$ is limited to 3, the output of the system cannot follow
the reference well.

\subsection{$\left| u(t)\right| \leq 10, \left| \Delta u(t) \right| \leq 3$, $T_s=1s$}

\begin{figure}[h!]
    \centering
    \includegraphics[width=0.7\textwidth]{pb1 4.png}
    \caption{Graph q5}\label{fig:pb2graph5}
\end{figure}
In the figure \ref{fig:pb2graph5}, the input signal is limited to 10,
and $\Delta u$ is limited to 3, the system baddly followed the reference.

\newpage
\section{Problem 3}
% insert 4 pictures here with caption
\begin{figure}[h!]
    \centering
    \includegraphics[width=0.7\textwidth]{graph a.png}
    \caption{Graph q1}\label{fig:pb3graph1}
\end{figure}
In the figure \ref{fig:pb3graph1}, the reward of the system is ascending,
the reward of the system is ascending. In the last episode the reward of
the system with bukets 5 is higher than the reward of the system with
bukets 10 and 2.

\begin{figure}[h!]
    \centering
    \includegraphics[width=0.7\textwidth]{graph b.png}
    \caption{Graph q2}\label{fig:pb3graph2}
\end{figure}
In the figure \ref{fig:pb3graph2}, the reward of the system is ascending,
the system with learnig rate 0.5 and 0.1 is ascending faster than the system with
learning rate 0.9.

\begin{figure}[h!]
    \centering
    \includegraphics[width=0.7\textwidth]{graph c.png}
    \caption{Graph q3}\label{fig:pb3graph3}
\end{figure}
In the figure \ref{fig:pb3graph3}, the reward of the system is fluxuating and slightly ascending,
system with discount factor 0.9 is ascending faster than discount factor 0.5, the system with
discount factor 0.5 is ascending faster than discount factor 0.1.

\begin{figure}[h!]
    \centering
    \includegraphics[width=0.7\textwidth]{graph d.png}
    \caption{Graph q4}\label{fig:pb3graph4}
\end{figure}
In the figure \ref{fig:pb3graph4}, the reward of the system is ascending,
the system with epsilon 0.1 is ascending faster than the system with epsilon 0.9.

\end{document}